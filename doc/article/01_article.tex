\documentclass[a4paper,twoside]{article}

\usepackage{epsfig}
\usepackage{subcaption}
\usepackage{calc}
\usepackage{amsfonts}
\usepackage{amssymb}
\usepackage{amstext}
\usepackage{amsmath}
\usepackage{amsthm}
\usepackage{array}
\usepackage{booktabs}
\usepackage{cite}
\usepackage{enumitem}
\usepackage{graphicx}
\usepackage{hyperref}
\usepackage{multicol}
\usepackage{pslatex}
\usepackage{algorithm2e}
\usepackage{xcolor}
\usepackage{interval}
\usepackage{xr}

\usepackage[font=scriptsize]{caption}
\usepackage[bottom]{footmisc}

\usepackage{SCITEPRESS}


\graphicspath{ {./img/} }
\intervalconfig{
    soft open fences,
}
\externaldocument{02_appendix}
\newcommand{\appendixurl}{https://www.cs.ubbcluj.ro/~andrei.olar/article-data/er-metrics-library/02_appendix.pdf}

\begin{document}
    \title{PyResolveMetrics: A Standards-Compliant and Efficient Approach to Entity Resolution Metrics}
    \author{
        \authorname{Andrei Olar\sup{1}\orcidAuthor{0009-{}0006-{}7913-{}9276}}\authorname{Laura Dio\c san\sup{1}\orcidAuthor{0000-{}0002-{}6339-{}1622}}
        \affiliation{\sup{1}Faculty of Mathematics and Computer Science, Babe\c s-Bolyai University}
        \email{andrei.olar@ubbcluj.ro,~laura.diosan@ubbcluj.ro}
    }
    \keywords{Entity Resolution,Metrics,Library,Open Source,Standards-Compliant,Theoretical Model,Efficiency}

    \abstract{Entity resolution, the process of discerning whether multiple data
    refer to the same real-world entity, is crucial across various domains,
    including education.
    Its quality assessment is vital due to the extensive practical
    applications in fields such as analytics, personalized learning or academic
    integrity.
    With Python emerging as the predominant programming language in these areas,
    this paper attempts to fill in a gap when evaluating the qualitative
    performance of entity resolution tasks by proposing a novel consistent
    library dedicated exclusively for this purpose.
    This library not only facilitates precise evaluation but also aligns with
    contemporary research and application trends, making it a significant tool
    for practitioners and researchers in the field.}
    \onecolumn\maketitle\normalsize\setcounter{footnote}{0}\vfill

    \section{Introduction}\label{sec:introduction}
    The field of entity resolution (ER), integral to natural language processing and
    emerging technologies in education, is pivotal in understanding and linking
    data across multiple sources.
    Some definitions view it as identifying and linking data from multiple
    sources\cite{Qia17}.
    However, it's argued that this identification and linking is a more
    specialized process\cite{Tal11}.

    ER, also known as record linkage, data deduplication,
    merge-purge, named entity recognition, entity alignment, and entity
    matching, plays a significant role in educational communication and
    collaboration tools, facilitating the information exchange between parents,
    teachers and students.
    It can also be useful to track the progression of alumni, by linking various
    data sources to provide meaningful insights on the career trajectories of
    graduates.     
    It is a key component in AI literacy, particularly in those machine learning
    tasks that involve identifying how disparate pieces of information correlate
    to the same real-world entity.
    ER itself has numerous implementations that rely on machine
    learning and is important for AI literacy for that reason,
    too\cite{deepm2020}. 
    We recognize the importance of accurate and efficient ER in
    both individual learning outcomes and broader societal impacts.
    Enhancing research integrity through plagiarism detection or enabling
    personalized learning by tracking a student's preferences across learning
    domains might qualify as fields of study in their own right.     
    For example, the work of~\cite{chen2021careerexploration} points out the
    importance of University and professional information for career exploration.
    Their survey highlights that matching the educational offering to fit
    student goals leads to a higher chance of students developing successful
    careers.
    One can envision systems that automatically create educational offerings
    based on the profiles of students.
    In this scenario, ER has the role of automatically building the student
    profile from heterogeneous information sources.
    Another use case for automatically generated profiles might be career
    recommender systems, for example.
    The ideal outcome from using the information stored in these profiles would
    be finding the best possible match between educational offering and student
    aspirations.
    
    What if ER provides us with a misleading profile?
    At best, we realize this is the case, stop trusting the ER system and revert 
    to a state where we don't benefit from the information
    stored in the profiles generated through ER.
    At worst, we do not realize the system error and proceed career exploration
    based on misleading profiles.
    This leads to bad career choice recommendation and more severe risks related
    to wasted time, financial misfortune, professional dissatisfaction,
    stagnation in personal development or even health and relationship concerns.
    Adopting the right ER system is desirable for obtaining the
    initial benefits of making more informed decisions faster.
    ER systems should not be adopted and cannot be maintained without measuring
    the quality of their outcomes.
    
    In this context, the paper introduces a new Open Source library that is
    hosted in a \href{https://github.com/matchescu/py-resolve-metrics}{Git
    repository on GitHub}\cite{matchescu-er-metrics2023}.
    The library offers implementations of well known metrics for evaluating
    ER, contributing significantly to metrics and evaluation in
    educational technologies.
    Thus, this library could have a role in advancing tools and methodologies in
    the realm of education and technology by allowing a more informed process
    for developing tools that make use of ER. 
    The book \cite{manning2008} studies various methods
    for evaluating the performance of information retrieval
    systems that help in assessing how effective these systems are in searching,
    identifying, and retrieving relevant information from large datasets.
    The metrics revolve around the notion of relevant and
    irrelevant information that is retrieved by the system.
    It is asserted that what is relevant is stipulated in a ground truth which
    is dependant upon an information need.
    ER systems partly function as information retrieval systems, as they
    determine whether multiple data points refer to the same real-world
    entity.
    This capability to discern data identity within a context is the fundamental
    information need of any ER system.
    Is it therefore fitting to use information retrieval metrics for entity
    resolution?
    This seems to be the consensus drawn in the scientific literature as we
    shall see in the next section.
    Our library implements various information retrieval metrics adapted for ER.
    
    It's also important to acknowledge the distinct entity resolution models.
    The library sets itself apart by organizing metrics based on their
    compatibility with ER models, influenced by the underlying
    differences in data structures that are characteristic to each model.
    Special attention is given to interoperability and the seamless integration
    of the library into the Python programming language ecosystem.
    Its key features are: embracing an OpenSource licensing model, efficient
    implementation using state of the art libraries under a very popular
    platform, and a design that is agnostic to the ER implementation.
    
    After this introduction, we overview two existing mathematical models for
    ER which are widely used and still represent the state of the art.
    Then we go through other work that relates to this paper.
    Subsequently, we present the new library and pay special attention to the
    reasons for implementing it.
    We go over the metrics that are implemented, the technological and design
    choices that were made, an example of using the library and a performance
    evaluation of the functions implemented by the library.
    In the end we offer some conclusions and present aspects that require more
    work.

    \section{Entity Resolution Models}\label{sec:models}
    \textbf{Fellegi-Sunter Model.}
    In the late 1960s Ivan Fellegi and Alan Sunter wrote the seminal
    paper\cite{fs1969} for record linkage --- what would later be known as ER.
    To this day, their mathematical model based on probability theory is the
    most popular way of formalizing the ER problem.
    In this mathematical model, ER is a function that aids in 
    probabilistic decision making.    
    In this model of ER, the process primarily involves comparing
    data from two sources.
    The essential step is matching two items --- one from each source, after
    which a decision is made to categorize the match as a `link', `non-link', or
    `possible link'.
    Consequently, any matching algorithm under this model typically returns
    pairs of items from the original data sources, each tagged as one of these
    categories.
    However, in practical applications today, this process is often simplified
    to just returning a list of pairs labeled as `links'. 
    This intuitive explanation gives us the structure of the input we can expect
    when we use the Fellegi-Sunter ER model: an iterable sequence
    of pairs.

    The metrics that are implemented with the Fellegi-Sunter model of entity
    resolution in mind will accept iterable sequences of pairs as input where
    the ground truth and the result of the ER task are concerned.

    \noindent\textbf{Algebraic Model.}
    The algebraic model for ER, initially conceived for assessing
    information quality in large datasets\cite{tal2007algebraic}, was later
    refined to describe the ER process itself\cite{Tal11}.
    This model treats ER as an algebraic equivalence relation
    over a given input set, which can include data from as many original sources
    as necessary.     
    The unique aspect of this model lies in the characteristics of equivalence
    relations\cite{halmos1960naive}.
    These relations create partitions over the input set, with each partition
    component equivalent to an equivalence class of the relation\cite{Tal11}.
    Conversely, a partition over a set can also induce an equivalence relation.
    With this in mind, evaluating the outcome of an ER task
    becomes as easy as comparing two partitions: the partition that induces the
    ideal equivalence relation (the gold standard or ground truth) to the
    partition that is produced by the ER task.

    The library supports a few metrics for comparing partitions, all of which
    expect that a partition is represented as a list of sets.

    \section{Related Work}\label{sec:related}

    Measuring ER quality was a subject of interest ever since the
    first paper on the subject surfaced\cite{newcombe1959}.
    It speaks about accuracy and contamination similarly to the current
    notions of true and false positives.
    The fundamental theory of record linkage\cite{fs1969} offers a probabilistic
    approach to evaluating the success of an ER task.
    It suggests methods to affect the results through the selection of suitable
    thresholds for defining success and failure.
    It also provides mechanisms for properly weighting for independent
    probability variables.
    The literature expands on these techniques in subsequent
    papers\cite{winkler1990}.
    Some of the ER evaluation metrics that are a direct result of
    this theoretical foundation include match accuracy, match
    rate\cite{jaro1989advances}, error rate estimation, rate of clerical
    disambiguation\cite{winkler1990} or relative distinguishing
    power of matching variables\cite{winkler2014}.
    A lot of effort is spent on estimating and measuring the
    effectiveness of blocking techniques to reduce the input size of the data
    set used for evaluation purposes\cite{winkler1990,jaro1989advances}.
    Measuring ER performance was and remains a computationally
    intensive task.
    
    Concerns about using accuracy and match rate are also voiced\cite{Goga2015}.
    Thus we see a shift towards metrics used in the related field of information
    retrieval.
    The probabilistic model for ER aligns well with concepts such
    as true/false positives/negatives.
    Given the extensive history of using ground truths to assess entity
    resolution quality, there is a natural fit for using information retrieval
    quality metrics.
    Most literature on this topic focuses on using information retrieval metrics
    where the order in which results are retrieved is not
    relevant\cite{manning2008}.

    Besides the original statistical model for ER, other models
    have evolved from it or alongside it.
    The work of the InfoLab at Stanford on their Stanford Entity Resolution
    Framework\cite{Ben2009Swoosh} and that of the Center for Entity Resolution
    and Information Quality at the University of Arkansas in Little
    Rock\cite{tal2007algebraic} stand out.
    These models of ER also propose new metrics for evaluating
    ER quality\cite{Men10,Tal11}.

    There is ample coverage of the metrics used for ER in syntheses on the
    subject\cite{vldb2010,hitesh2012,Tal11}.
    Clustering metrics such as pairwise and cluster metrics\cite{Men10, huang2006efficient}
    or the Rand index\cite{tal2007algebraic} seem to be used more frequently to
    measure ER quality as time passes.

    Numerous systems to perform ER are available.
    Some of them include modules to evaluate the performance of a
    particular ER solution\cite{fever2009,magellan2020,oyster2012}.
    There are also other Python packages that implement some or all of the
    metrics provided by our library\cite{ereval,virtanen2020scipy}.

    \section{PyResolveMetrics}\label{sec:library}

    In this context, the necessity for yet another specialized library dedicated
    to evaluating ER metrics might seem redundant.
    This skepticism is rooted in the expectation that Python, being a highly
    popular programming platform, should already offer high-quality, reusable
    tools available for a wide range of applications --- including evaluating
    ER results.
    
    Upon closer examination of the tools available for evaluating entity
    resolution tasks, certain limitations in the existing assumptions become
    apparent.
    There are indeed numerous libraries offering packages for computing entity
    resolution metrics.
    However, using a general-purpose library like SciPy raises concerns about
    interoperability and efficiency.
    This is particularly relevant when the sole requirement is to compute entity
    resolution metrics, and the additional features of a comprehensive library
    are unnecessary.
    The challenge of seamlessly integrating ER evaluation
    routines into a custom built project becomes even more pronounced when
    attempting to use the ones packaged with established ER
    systems~\cite{oyster2012},~\cite{jedai2017},~\cite{deepm2020},~\cite{magellan2020}.
    
    Conversely, when specifically searching for libraries that only offer ER
    metrics, it becomes evident that some of the essentials for effectively
    evaluating ER tasks may be absent\cite{ereval}.

    Approaching the issue from a different angle, using metrics from a
    general-purpose algorithmic library like Scipy (specifically
    \texttt{scipy.metrics}) for ER evaluation requires strict
    adherence to certain design choices imposed by the library.
    For example, to calculate the Rand index, data clusters must be mapped with
    labels, and these labels must be provided as input.
    While this might seem simple, the user-friendliness of such an approach is
    debatable.
    The complexity of adapting existing data and managing the necessary labels
    for the package could potentially rival the complexity of computing the Rand
    index itself, mooting the use of the package.
    Furthermore, additional memory and compute time are also required to perform
    the mapping between own data structures and the ones required by the API
    contract.

    In short, here are the reasons we chose to implement such a library:
    \begin{itemize}
    \item Architecturally, adhering to the principle of `do one thing and do
    it well' is beneficial.
    This approach avoids the biases and dependencies of general-purpose
    libraries like SciPy, which can complicate integration into our
    custom-designed software.
    \item Historically, ER has adapted evaluation techniques from
    statistics, information retrieval, and graph theory, tailoring these methods
    to suit its specific needs.
    It seems desirable to standardize these methods into forms specific to
    ER.
    \item Currently, there appears to be no implementation that consolidates all
    the metrics useful for ER evaluation, as identified in
    scientific literature, into a single, cohesive unit.
    \item Our work has a significant component of evaluating ER
    outcomes.
    \end{itemize}

    Our opinion is that using mathematical models specific to ER
    is the best approach for guiding the library's design.
    Since each model significantly impacts the data structures used in
    evaluation, the library's functions are categorized based on the type of
    input they support and, implicitly, by the mathematical model they align
    with. 
    There are a couple of important assumptions that the library makes,
    regardless of the ER model.
    One such assumption is that the quality of the ER output is
    always measured against a ground truth\cite{manning2008}.
    The other assumption is that the ground truth and the ER
    result are both structured under the same mathematical model.

    \subsection{Supported Metrics}\label{sec:metrics}
    \textbf{Statistical quality metrics}, extensively detailed in the
    literature\cite{manning2008,hitesh2012}, are the most common method for
    measuring ER performance as evidentiated by their almost
    ubiquitous usage\cite{fever2009,Goga2015,deepm2020,eager2021}.
    These metrics are linked to the Fellegi-Sunter model for ER
    which provides clear definitions of Type I and Type II
    errors\cite{winkler1990}.
    Type I and Type II errors clarify the concepts of true positives, true
    negatives, false positives, and false negatives as they are used in entity
    resolution.
    Understanding these concepts necessitates referencing the $M$ (matches) and
    $U$ (non-matches) sets as defined in the seminal paper on the model.

    Depending on the expected location of a pair produced by the entity
    resolution function, we define:

    \begin{itemize}
        \item \textbf{true positives} as pairs predicted to be in $M$ that
        should be in $M$,
        \item \textbf{false positives}, or type I errors, as pairs predicted to
        be in $M$, but should be in $U$,
        \item \textbf{true negatives} as pairs predicted to be in $U$ that
        should be in $U$, and
        \item \textbf{false negatives}, or type II errors, as pairs predicted to
        be in $U$, but should be in $M$.
    \end{itemize}

    Several metrics based on these concepts exist, though the effectiveness of
    some has been questioned\cite{Goga2015}.
    With this in mind we finally define the three quality metrics that are
    supported by our library:

    \begin{align}
    Precision&=\frac{true\,positives}{true\,positives + false\,positives} \\
    Recall&=\frac{true\,positives}{true\,positives + false\,negatives} \\
    F_1 Score&=2 \cdot \frac{Precision \cdot Recall}{Precision+Recall}
    \end{align}

    \textit{Precision} (or the positive predictive value) is defined as the
    number of correct predictions that were made in relation to the total number
    of predictions that were made.
    \textit{Recall} (or sensitivity) is defined as the number of correct
    predictions that were made in relation to the total number of positive
    predictions that could have been made (which corresponds to the number of
    items in the ground truth).
    The \textit{$F_1$} score is the harmonic mean of the precision and the
    recall and it is used to capture the tradeoff between precision and
    recall\cite{hitesh2012}.

    \textbf{Algebraic metrics} is the generic name we use for `cluster metrics'\cite{rand1971,hitesh2012}
    and `pairwise metrics'\cite{hitesh2012,Men10} because their foundation is
    algebraic and because they are linked to the algebraic model for ER.
    Most of the algebraic metrics implemented by the library are an exercise in
    using operations on sets, while the rest focus on matrix operations with a
    dash of combinatorics:
    \begin{itemize}
        \item Pairwise metrics (precision, recall and F-measure)\cite{Men10,hitesh2012}
        \item Cluster metrics (precision, recall and F-measure)\cite{huang2006efficient,hitesh2012}
        \item Talburt-Wang Index\cite{tal2007algebraic}
        \item Rand\cite{rand1971} and Adjusted Rand Index\cite{adjrand1985}
    \end{itemize}
    Their input arguments (the ground truth and the ER result) are represented
    as partitions over \textit{the same} set.
    
    The Rand index is one of the first metrics used to compare the similarity
    between two different data clusterings.
    It quantifies the agreement or disagreement between these clusterings by
    considering pairs of elements.
    \begin{equation}
        Rand Index = \frac{(a + b)}{{\binom{n}{2}}}
    \end{equation}
    The main components of the Rand index are as follows:
    a: Represents the number of times a pair of elements belongs to the
        same cluster across both clustering methods, 
    b: Represents the number of times a pair of elements belongs to
        different clusters across both clustering methods,
    $\binom{n}{2}$: denotes the number of unordered pairs in a set of n
        elements.
   
    The Rand index always takes values in the $\rinterval{0}{1}$ interval.

    A variation on the Rand Index is the Adjusted Rand Index for chance grouping
    of elements.
    It accounts for agreements between data clusterings that occur due to
    chance~\cite{adjrand2001}.
    The Adjusted Rand Index is calculated by using the following formula:
    \begin{equation}
        ARI = \frac{RandIndex - E}{\max(RandIndex) - E},
    \end{equation}
    \noindent
    where $E$ is the expected value of the RandIndex.
    The Adjusted Rand index is valued in the interval $\interval{-1}{1}$.
    For a comprehensive understanding of the Adjusted Rand Index and its
    calculation, we recommend consulting the detailed and informative work
    presented in the study by~\cite{warrens2022understanding} on the subject. 
    For both of these indexes, higher scores indicate a closer alignment between
    the compared partitions.

    A metric that attempts to approximate the Rand Index is the Talburt-Wang
    Index which counts the number of overlapping subsets of two partitions over
    the same input set.
    Assuming $A$ and $B$ are two partitions over the same input set of elements,
    the Talburt-Wang Index is given by the formula:
    \begin{equation}
        \varDelta(A, B) = \frac{|A|\cdot|B|}{\varPhi{\ointerval{A}{B}}^2}
    \end{equation}
    \noindent 
    where $\varPhi(A, B) = \sum_{i=1}^{|A|}\{B_j \in B | B_j \cap A_i \neq \emptyset \}$.

    This metric approximates the Rand Index without requiring the expensive
    counting of true positives, false positives, true negatives or false
    negatives~\cite{tal2007algebraic}.
    It is valued within the same interval as the Rand Index.
    
    Our library implements other popular metrics that can be used for comparing
    partitions: pairwise precision, pairwise recall and their harmonic
    mean (the pairwise $F_1$ measure)\cite{hitesh2012}.
    
    If we have two sets $X$ and $Y$, the pairwise precision is given by the
    ratio of pairs that are in both sets to the total amount of pairs of the
    reference set.
    \begin{equation}
        PP(X, Y) = \frac{|{Pairs(X)}\cap{Pairs(Y)}|}{|Pairs(X)|}
    \end{equation}    
    The pairwise recall is given by the ratio of pairs that are in both sets
    to the number of pairs in the comparison set\cite{hitesh2012}.
    \begin{equation}
        PR(X, Y) = \frac{|{Pairs(X)}\cap{Pairs(Y)}|}{|Pairs(Y)|}
    \end{equation}
    The pairwise F-measure is given by the harmonic mean of the pairwise
    precision and pairwise recall.
    \begin{equation}
        PF = \frac{2 \cdot PP \cdot PR}{PP + PR}
    \end{equation}
    The library computes partition metrics by iteratively analyzing equivalence
    classes within each partition generated by the ER equivalence relation and
    extracting element pairs from each subset.
    
    Finally, our library supports `cluster measures'\cite{hitesh2012}.
    Cluster precision is the ratio of the number of completely correct
    clusters to the total number of clusters resolved, whereas cluster recall
    is the portion of true clusters resolved\cite{huang2006efficient}.
    The harmonic mean of the cluster precision and cluster recall is typically
    called the cluster F-measure.
    In this paragraph `clusters' refer to the equivalence classes of the entity
    resolution relation as it is formalized in the algebraic model.
    
    Given two partitions $A$ and $B$, the cluster measures are given by the
    following formulae:

    \begin{align}
        CP(A, Y) &= \frac{|{A}\cap{B}|}{|A|}\\
        CR(A, Y) &= \frac{|{A}\cap{B}|}{|B|}\\
        CF &= \frac{2\cdot{CP}\cdot{CR}}{CP + CR}
    \end{align}

    \subsection{Technology}\label{sec:tech}

    The technology used for implementing our library is described in
    Section~\ref{appendix:sec:technology} of the Appendix available \href{\appendixurl}{online}.
    
    \subsection{Example Usage}

    To provide a visual outlook over the metrics provided by our library we are
    using a \href{https://github.com/matchescu/experiment-data}{toy data set}\cite{expdata2023}
    containing near duplicates and the PPJoin\cite{ppjoin} entity matching
    algorithm to perform ER.
    The PPJoin algorithm matches items by using prefix lengths determined using
    a Jaccard coefficient $t$.
    
    We split the data in the toy data set into two data sets by column.
    The resulting data sets are:
    
    \begin{enumerate}[label={\bfseries DG\arabic*:},leftmargin=1cm]
        \item with \texttt{`name', `manufacturer', `price', `id'},
        and
        \item with \texttt{`description', `name', `id'}\@
    \end{enumerate}

    Because we have split the data column-wise, we know exactly what the ground
    truth should be for each of the metrics, assuming that each row in the
    original toy data set refers to a distinct real-world entity.
    Because we are working with two data sets, the ground truth for the
    statistical model is the same as the ground truth for the algebraic model:
    a list of pairs of matching items obtained by iterating over DG1 and DG2
    using the same cursor.

    All that's left is to apply the PPJoin algorithm on DG1 and DG2 and plot the
    values of the metrics provided by the library for values of $t$ in the
    interval $\rinterval{0}{1}$ at increments of $0.01$.
    The plots that show the outcome are available~\href{\appendixurl}{online} in
    Section~\ref{appendix:sample-figures} of the accompanying Appendix.

    \subsection{Performance}

    CPU performance is usually evaluated by throughput (e.g~ millions of
    operations per second or MIPS).
    However it is meaningless to compare throughput on different CPU
    architectures~\cite{jain1991profiling}.

    A similar concern can be raised about memory profiling in relation to the
    underlying operating system.
    Due to the variability of the outcomes during experimentation and the fact
    that all memory consumption is very dependent on the operating system and
    standard C library used for compiling and linking the Python interpreter,
    we found memory profiling not to provide great insights.

    Under these circumstances we have chosen to elaborate a method of profiling
    the CPU usage of the library which is agnostic to the underlying hardware.
    CPU profiling is useful in the context of judging the metrics provided by
    the library relatively to one another.

    To prevent the ER task from interfering with profiling the metrics library,
    we run our experiment in two stages.
    In the first stage we run the ER task and store its result in a file along
    with the ground truth.
    The second stage loads the results and ground truth from the file and runs
    the entity resolution metrics while profiling the CPU usage.

    The performance analysis is available \href{\appendixurl}{online} in Section
    \ref{appendix:sec:perf} of the Appendix to this article.
    Perhaps the most important lesson to learn from profiling our library is
    that algebraic metrics are an order of magnitude more expensive to compute
    than statistical metrics.
    Moreover, not all algebraic metrics were created equal: the Rand indexes
    are an order of magnitude more expensive to compute than the other algebraic
    metrics.
    Therefore, because the Talburt-Wang index approximate the Rand index well,
    it might be a preferable choice to measure how well an ER algorithm performs
    clustering.

    \section{Conclusions and Future Work}\label{sec:conclusions_and_future}

    We have introduced a library for evaluating ER results that
    is based on standards and Python protocols, making it highly interoperable.
    The API exposed by this library is deeply rooted in the mathematical models
    fundamental to ER, making it more familiar to ER users.

    The performance of the library is sound because it externalizes
    computationally expensive tasks to native code.
    The accuracy of the implemented metrics is verified automatically through
    unit tests.

    These attributes render the library not only highly beneficial but also low
    maintenance, making it an invaluable asset ER.\@
    This does not preclude additional work.

    \textbf{Missing metrics.}
    The ER models we have touched upon support many more metrics that the
    library does not currently implement.
    The work by~\cite{hitesh2012} provides an insightful overview.
    For well-rounded support of the ER models mentioned so far, the library
    should implement at least additional cluster comparisons, such as the
    Closest Cluster $F_1$, the MUC $F_1$, $B^3 F_1$ and the
    $CEAF F_1$~\cite{hitesh2012}, and additional Rand-like indexes~\cite{warrens2022understanding}.

    \textbf{Missing models.} 
    Besides the models we have covered herein, ER has been
    theorized to be a graph problem\cite{eager2021} or an exercise in lattice
    theory with an ordering relation based on
    ``merge dominance''\cite{Ben2009Swoosh}.
    More work is required to distil the metrics that become available for
    evaluating ER under those models and the data structures that are used in
    the evaluation process.

    \bibliographystyle{apalike}
    {
        \small
        \bibliography{03_references}
    }
\end{document}
